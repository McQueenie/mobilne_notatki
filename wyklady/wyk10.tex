\section*{Wykład 10 - Testowanie aplikacji mobilnych}
\textit{12.05.2025 - Dr inż. Jakub Długosz}

\section{Środowiska programistyczne}

\begin{tcolorbox}[colback=blue!5!white,colframe=blue!75!black,title=Podział środowisk]
\begin{itemize}[leftmargin=*]
    \item \textbf{Deweloperskie} -- przeznaczone do bieżącej pracy nad rozwojem oprogramowania
    \item \textbf{Produkcyjne} -- dedykowane końcowemu użytkownikowi
    \item \textbf{Testowe} -- środowiska pośrednie umożliwiające wykonywanie testów
\end{itemize}
\end{tcolorbox}

\section{Klasyfikacja testów oprogramowania}

\subsection{Podstawowe rodzaje testów}

\begin{enumerate}
\item \textbf{Testy jednostkowe}
    \begin{itemize}
        \item Testują jednostkę kodu (klasę, obiekt, funkcję) w izolacji
        \item Zależności symulowane przez mocki lub stuby
        \item Cel: znalezienie błędów w implementacji danego komponentu
    \end{itemize}

\item \textbf{Testy integracyjne}
    \begin{itemize}
        \item Testują współpracę między komponentami
        \item Zakres: od kilku klas do różnych systemów (baza danych, serwery)
        \item Cel: wykrycie błędów podczas interakcji między systemami
    \end{itemize}

\item \textbf{Testy systemowe}
    \begin{itemize}
        \item Przeprowadzane po integracji elementów systemu
        \item Obejmują testy end-to-end (E2E) z perspektywy użytkownika
        \item Podstawa: wymagania, przypadki użycia, specyfikacja
    \end{itemize}

\item \textbf{Testy akceptacyjne}
    \begin{itemize}
        \item Wykonywane przez klienta lub użytkowników końcowych
        \item Cel: upewnienie się, że aplikacja spełnia oczekiwania klienta
    \end{itemize}
\end{enumerate}

\subsection{Testy specjalistyczne}

\begin{itemize}
    \item \textbf{Testy wydajnościowe} -- sprawdzają wydajność systemu (Apache Bench, JMeter)
    \item \textbf{Testy bezpieczeństwa} -- analiza pod kątem bezpieczeństwa (Pentesty, OpenVAS)
    \item \textbf{Testy smoke} -- podstawowe testy głównych funkcji systemu
    \item \textbf{Testy eksploracyjne} -- swobodne testowanie bez ścisłego scenariusza
\end{itemize}

\section{Modele organizacji testów}

\begin{tcolorbox}[colback=green!5!white,colframe=green!75!black]
\textbf{Piramida testów} $\rightarrow$ \textbf{Model Kryształ} $\rightarrow$ \textbf{Model Trofeum}
\begin{itemize}
    \item Różne podejścia do proporcji rodzajów testów
    \item Piramida: dużo testów jednostkowych, mało E2E
    \item Trofeum: więcej testów integracyjnych
\end{itemize}
\end{tcolorbox}

\section{Testowanie manualne vs automatyczne}

\begin{tabular}{|p{0.45\textwidth}|p{0.45\textwidth}|}
\hline
\textbf{Testowanie manualne} & \textbf{Testowanie automatyczne} \\
\hline
- Wykonywane osobiście przez testera & - Wykonywane przez maszynę \\
- Kosztowne czasowo & - Szybkie wykonanie \\
- Podatne na błędy ludzkie & - Powtarzalne i niezawodne \\
- Elastyczne & - Wymaga napisania skryptów \\
\hline
\end{tabular}

\section{Specyficzne aspekty testowania aplikacji mobilnych}

\begin{enumerate}[leftmargin=*]
    \item \textbf{Przenoszenie preferencji systemowych} (np. wyciszenie dźwięków)
    \item \textbf{Reakcja na przerwania} (połączenia telefoniczne)
    \item \textbf{Zarządzanie baterią} (zachowanie przy niskim stanie baterii)
    \item \textbf{Uprawnienia aplikacji} (reakcja na wyłączenie przywilejów)
    \item \textbf{Responsywność} (różne urządzenia, rozdzielczości, orientacje ekranu)
    \item \textbf{Połączenie internetowe} (brak/zmiana typu połączenia)
\end{enumerate}

\section{Narzędzia i platformy testowania}

\subsection{Farmy urządzeń w chmurze}
\begin{itemize}
    \item AWS Device Farm
    \item Firebase Test Lab (10 testów/dzień na wirtualnych, 5 na fizycznych - plan darmowy)
    \item BrowserStack
    \item Azure DevOps Pipelines
    \item LambdaTest
\end{itemize}

\subsection{Automatyzacja testów}
\textbf{Appium} -- popularne narzędzie do automatyzacji testów aplikacji mobilnych
\begin{itemize}
    \item Wspiera Android i iOS
    \item Pozwala pisać testy w różnych językach programowania
    \item Dokumentacja: \url{https://appium.io/docs/en/latest/}
\end{itemize}

\section{ISTQB CT-MAT}
Certyfikacja dla testerów aplikacji mobilnych obejmująca:
\begin{itemize}
    \item Specyfikę testowania mobilnego
    \item Internacjonalizację (i18n) i lokalizację (l10n)
    \item Testowanie na różnych platformach i urządzeniach
\end{itemize}

