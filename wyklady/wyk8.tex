\section*{Wykład 8 - UI/UX i Interfejsy Mobilne}
\textit{28.04.2025 - Systemy mobilne i multimedia}

\section{Interfejs użytkownika}

\textbf{Definicja}: System wspierający interakcję człowiek-komputer lub człowiek-maszyna, składający się z:
\begin{itemize}
    \item Oprogramowania (software)
    \item Sprzętu (hardware) 
    \item Wejść (sterowanie systemem)
    \item Wyjść (odpowiedź systemu)
\end{itemize}

\textbf{Cechy interfejsów multimedialnych}:
\begin{itemize}
    \item Interaktywność -- komunikacja dwukierunkowa
    \item Użytkownik może kształtować treści (zakres, ilość, format)
    \item Może być współtwórcą treści
\end{itemize}

\section{Rodzaje interfejsów użytkownika}

\begin{enumerate}
    \item \textbf{Wiersz polecenia} -- interakcja przez wpisywanie poleceń
    \item \textbf{Graficzne (GUI)} -- tekst + ikony + pomoce wizualne
    \item \textbf{Dotykowe} -- single-touch lub multi-touch
    \item \textbf{Głosowe} -- Siri, Asystent Google, Copilot, Alexa
    \item \textbf{Wirtualnej rzeczywistości (VR)} -- gogle + kontrolery
    \item \textbf{Rozszerzonej rzeczywistości (AR)} -- warstwy informacji na świecie realnym
\end{enumerate}

\section{Słowniczek pojęć}

\begin{itemize}
    \item \textbf{UX} -- User Experience (wrażenia użytkownika)
    \item \textbf{UI} -- User Interface (interfejs użytkownika)
    \item \textbf{GUI} -- Graphical User Interface
    \item \textbf{HCI} -- Human-Computer Interaction
    \item \textbf{HCD} -- Human Centered Design
    \item \textbf{UCD} -- User Centered Design
    \item \textbf{UDD} -- User Driven Development
\end{itemize}

\section{Użyteczność wg Jakoba Nielsena}

Podstawowa miara jakości interfejsu, składająca się z:

\begin{enumerate}
    \item \textbf{Zdolność nauczenia się} -- jak szybko użytkownik opanuje nowy interfejs
    \item \textbf{Efektywność} -- szybkość realizacji zadań po nauczeniu
    \item \textbf{Zdolność zapamiętania} -- łatwość odtworzenia po przerwie
    \item \textbf{Reakcja na błędy} -- eliminacja, ograniczanie, łatwa naprawa
    \item \textbf{Poziom satysfakcji} -- radość z korzystania
\end{enumerate}

\section{Personalizacja vs Dostosowywanie}

\subsection{Personalizacja (personalization)}
\begin{itemize}
    \item System wykrywa potrzeby użytkownika
    \item Automatyczne dostosowanie
    \item Uczenie maszynowe
    \item Potrzeby niesformułowane wprost
\end{itemize}

\subsection{Dostosowywanie (customization)}
\begin{itemize}
    \item Użytkownik sam zmienia opcje
    \item Motywy, kolory, czcionki
    \item Podział na kosmetyczne i funkcjonalne
    \item Świadomy wybór użytkownika
\end{itemize}

\section{I18n vs L10n}

\begin{itemize}
    \item \textbf{I18n} (Internationalization) -- internacjonalizacja
    \item \textbf{L10n} (Localization) -- lokalizacja
\end{itemize}

Tworzenie oprogramowania dostępnego dla użytkowników z różnych regionów świata.

\section{Material Design (Android)}

\subsection{Kluczowe elementy}
\begin{itemize}
    \item \textbf{Jednostki}: dp (density-independent pixels), sp (scale-independent pixels)
    \item \textbf{Zasady dostępności}:
    \begin{itemize}
        \item Doceniaj jednostki (honor individuals)
        \item Ucz się na wstępie (learn before, not after)  
        \item Wymagania jako punkt wyjścia
    \end{itemize}
\end{itemize}

\subsection{Komponenty UI}
\begin{itemize}
    \item Działania (Actions)
    \item Komunikacja (Communication)
    \item Ograniczające (Containment)
    \item Nawigacja (Navigation)
    \item Wybór (Selection)
    \item Pola tekstowe (Text inputs)
\end{itemize}

\subsection{Ważne aspekty}
\begin{itemize}
    \item \textbf{Gesty} -- standardowe wzorce interakcji
    \item \textbf{Zmiana stanu} -- warstwy stanów (state layers)
    \item \textbf{Kolor i kontrast} -- dostępność cyfrowa
\end{itemize}

\section{Apple Human Interface Guidelines (iOS)}

\subsection{Najlepsze praktyki dla gestów}
\begin{itemize}
    \item Zapewnij wiele sposobów interakcji
    \item Zachowaj spójność z innymi aplikacjami
    \item Definiuj własne gesty tylko gdy konieczne
    \item Reaguj responsywnie
    \item Oferuj skróty gestowe jako uzupełnienie
\end{itemize}

\subsection{Tryb pełnoekranowy}
\begin{itemize}
    \item Wspieraj gdy ma to sens
    \item Zachowaj dostęp do kluczowych funkcji
    \item Pozwól użytkownikowi wybrać moment wyjścia
    \item Umożliw dostęp do Docka
\end{itemize}

\section{Definicja użyteczności wg ISO 9241}

\textbf{Użyteczność} = efektywność + wydajność + satysfakcja

\begin{itemize}
    \item \textbf{Efektywność}: dokładność i pełność osiągania celów
    \item \textbf{Wydajność}: zasoby zużyte względem dokładności
    \item \textbf{Satysfakcja}: komfort i akceptowalność systemu
\end{itemize}

\section{Historia UX -- kluczowe momenty}

\begin{itemize}
    \item \textbf{XIX w.} -- Frederick Taylor, tayloryzm
    \item \textbf{Lata 40. XX w.} -- System Produkcyjny Toyoty
    \item \textbf{1955} -- Henry Dreyfuss ,,Designing for People''
    \item \textbf{Lata 70.} -- Xerox PARC: GUI, mysz, format bitmap
    \item \textbf{Lata 90.} -- Donald Norman formułuje termin ,,user experience''
\end{itemize}

\section{Wybrane zasady projektowe Dona Normana}

\begin{enumerate}
    \item \textbf{Projektuj dla prostoty} -- uwzględnij złożoność, ale produkt powinien być zrozumiały
    \item \textbf{Projektuj dla prawdziwych ludzi} -- nie dla ideałów
    \item \textbf{Nie bądź logiczny} -- zwykli ludzie nie myślą abstrakcyjnie jak matematycy
    \item \textbf{Błędy to cenne doświadczenia} -- uczenie się na błędach
    \item \textbf{Więcej się uczymy z błędów niż sukcesów} -- konstruktywna krytyka poprawia projekty
\end{enumerate}

\section{Przedmiot Normana}

Obiekt, którego sposób użycia jest odwrotny niż oczekiwany na podstawie wyglądu (np. drzwi Normana). Korzystanie przeczy intuicji użytkownika.

