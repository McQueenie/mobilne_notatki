\section*{Wykład 6 - Sygnały, Informacja i Kompresja}
\textbf{Data:} 7.04.2025 \\
\textbf{Wykładowca:} dr inż. Jakub Długosz

\section{Sygnały}

\subsection{Definicja}
\textbf{Sygnał} -- nośnik informacji odzwierciedlający zmianę stanu obiektu fizycznego lub mierzalnej wielkości fizycznej.

\subsection{Szum}
\textbf{Szum} -- niepożądane składniki sygnałów.

\subsubsection{Rodzaje szumów}
\begin{itemize}[noitemsep]
    \item \textbf{Wewnętrzne (procesowe)} -- związane z obserwowanym obiektem
    \item \textbf{Zewnętrzne (zakłócenia)} -- niezwiązane z obserwowanym obiektem
\end{itemize}

\subsubsection{Typy szumów}
\begin{itemize}[noitemsep]
    \item \textbf{Szum biały} -- płaskie widmo, równomierna intensywność w całym paśmie, brak pamięci
    \item \textbf{Szum kolorowy} -- nierównomierny rozkład widmowy mocy (czerwony/Browna, różowy, niebieski, fioletowy)
    \item \textbf{Szum impulsowy} -- krótkotrwałe impulsy o dużej amplitudzie
\end{itemize}

\section{Model Komunikacji Shannona-Weavera (1948)}

Model liniowy z jednokierunkowym przepływem informacji:
\begin{center}
Źródło $\rightarrow$ Nadajnik $\rightarrow$ Kanał (+szum) $\rightarrow$ Odbiornik $\rightarrow$ Cel
\end{center}

\section{Przetwarzanie Analogowo-Cyfrowe}

\subsection{Podstawowe procesy}
\begin{enumerate}[noitemsep]
    \item \textbf{Próbkowanie} -- modulacja amplitudowa (AM)
    \item \textbf{Kwantowanie} -- przyporządkowanie wartościom rzeczywistym wartości dyskretnych
    \item \textbf{Kodowanie} -- reprezentacja cyfrowa
\end{enumerate}

\section{Informacja}

\textbf{Definicja}: To wszystko, co przekazane okazuje się użyteczne dla odbiorcy. Służy realizacji celu, zaspokaja potrzeby, buduje wiedzę.

\subsection{Cechy przekazu}
\begin{itemize}[noitemsep]
    \item Odbiorca weryfikuje użyteczność danych
    \item Dane mają znaczenie opisane funkcją semantyczną
    \item Forma przekazu równie ważna jak treść
\end{itemize}

\section{Kompresja Danych}

\subsection{Definicje}
\textbf{Kompresja} -- odwracalny lub nieodwracalny proces redukcji długości reprezentacji danych.

\subsection{Rodzaje kompresji}
\begin{enumerate}
    \item \textbf{Bezstratna (odwracalna)}
    \begin{itemize}[noitemsep]
        \item Rekonstrukcja z dokładnością do pojedynczego bitu
        \item Zastosowanie: dokumenty tekstowe, dane finansowe, niektóre obrazy medyczne
    \end{itemize}
    
    \item \textbf{Stratna (nieodwracalna)}
    \begin{itemize}[noitemsep]
        \item Brak możliwości dokładnej rekonstrukcji
        \item Pojęcie bezstratności percepcyjnej (wizualnej, słuchowej)
        \item Zastosowanie: multimedia (obrazy, dźwięk, wideo)
    \end{itemize}
\end{enumerate}

\subsection{Miary efektywności}
\begin{itemize}[noitemsep]
    \item \textbf{CR (Compression Ratio)} -- stosunek bitów oryginalnych do skompresowanych (np. 100:1)
    \item \textbf{CP (Compression Percentage)} -- $CP = (1 - 1/CR) \cdot 100\%$
    \item \textbf{BR (Bit Rate)} -- średnia liczba bitów na element źródłowy
\end{itemize}

\section{Metody Kodowania}

\subsection{RLE (Run Length Encoding)}
Kodowanie długości sekwencji -- seria powtórzeń symboli opisywana parą: (długość, symbol).

\subsection{Twierdzenie Shannona}
Aby zakodować proces o entropii $H(S)$ do postaci binarnej z możliwością dokładnej rekonstrukcji, potrzeba co najmniej $H(S)$ bitów.

\section{Standardy Kompresji}

\subsection{JPEG}
\begin{itemize}[noitemsep]
    \item Konwersja RGB $\rightarrow$ YCrCb
    \item Wykorzystuje DCT (Discrete Cosine Transform)
    \item Kompresja stratna dla obrazów
\end{itemize}

\subsection{MPEG}
Grupa standardów dla kompresji wideo i audio.

\section{Strumieniowanie}
\textbf{Definicja}: Przesyłanie danych w formie strumienia z wykorzystaniem kolejno napływających danych bezpośrednio po otrzymaniu.

\section{Metadane}
\begin{itemize}[noitemsep]
    \item \textbf{Deskryptor} -- numeryczny sposób opisu atrybutów obiektów multimedialnych
    \item \textbf{Indeks} -- zbiór cech/wartości atrybutu wraz z identyfikatorami obiektów
\end{itemize}

