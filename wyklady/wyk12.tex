\section*{Wykład 12 - Grafika 3D - Blender część 1}
\textit{26.05.2025 - Notatki z wykładu SMiM}

\section{Historia Blendera}
\begin{itemize}[itemsep=2pt]
    \item \textbf{2002} - firma NaN miała problemy finansowe
    \item Ton Roosendaal powołał \important{Blender Foundation}
    \item Zebrano 100 000 EUR z dobrowolnych wpłat
    \item Blender stał się \important{open source} (GNU GPL)
    \item Konkurencja: Maya, 3ds Max (oprogramowanie komercyjne)
\end{itemize}

\section{Podstawowe elementy sceny}
\begin{enumerate}
    \item \important{Kamera} - określa perspektywę renderowania
    \item \important{Światło} - źródło oświetlenia sceny
    \item \important{Obiekt} - domyślnie sześcian/kostka
\end{enumerate}

\section{Nawigacja w przestrzeni 3D}
\begin{itemize}[itemsep=2pt]
    \item \textbf{Orbitowanie}: \key{ŚKM} + przesuwanie
    \item \textbf{Zoom}: rolka myszy, \key{Ctrl +}, \key{Ctrl -}
    \item \textbf{Panoramowanie}: \key{Shift + ŚKM} + przesuwanie
\end{itemize}

\section{Najważniejsze skróty klawiszowe}
\begin{center}
\begin{tabular}{|l|l|}
\hline
\textbf{Skrót} & \textbf{Funkcja} \\
\hline
\key{Tab} & Przełączanie tryb obiekt $\leftrightarrow$ tryb edycji \\
\key{Ctrl Tab} & Wybór trybu siatki (wierzchołki/krawędzie/ściany) \\
\key{G} & Przesuwanie (Grab) \\
\key{R} & Obracanie (Rotate) \\
\key{S} & Skalowanie (Scale) \\
\key{G G} & Edge slide (przesuwanie wzdłuż krawędzi) \\
\key{Z} & Przełączanie widoku (szkielet/solidny) \\
\key{Shift S} & Menu ustawiania kursora \\
\key{F12} & Renderowanie obrazu \\
\key{Ctrl F12} & Renderowanie animacji \\
\key{I} & Utworzenie klatki kluczowej (animacja) \\
\hline
\end{tabular}
\end{center}

\section{Tryby pracy}
\begin{enumerate}
    \item \textbf{Tryb obiektu} - operacje na całych obiektach
    \item \textbf{Tryb edycji} - modyfikacja geometrii (wierzchołki, krawędzie, ściany)
\end{enumerate}

\section{Operacje modelowania}
\begin{itemize}[itemsep=2pt]
    \item \textbf{Loop Cut} - dodawanie przekrojów poprzecznych
    \item \textbf{Subdivide} - zagęszczanie siatki
    \item \textbf{Extrude} - wyciąganie geometrii
    \item \textbf{Push/Pull} - deformacja struktury siatki
\end{itemize}

\section{Materiały i powierzchnie}
\begin{itemize}[itemsep=2pt]
    \item Domyślny materiał: \important{Principled BSDF}
    \item BSDF = Bidirectional Scattering Distribution Function
    \item \textbf{Materiał} - określa wygląd powierzchni o stałej strukturze
    \item \textbf{Tekstura} - określa wygląd powierzchni niejednorodnej (wzory, zagłębienia)
\end{itemize}

\section{Światło - podstawowe właściwości}
\begin{enumerate}
    \item \textbf{Emisja} - współczynnik emisji światła
    \item \textbf{Odbicie} - współczynnik odbicia
    \item \textbf{Pochłanianie} - współczynnik absorpcji
\end{enumerate}

\section{Renderowanie}
\begin{itemize}[itemsep=2pt]
    \item \important{Renderowanie} - tworzenie obrazów z perspektywy kamery
    \item Uwzględnia: oświetlenie sceny, właściwości fizyczne powierzchni
    \item Dostępne silniki: Cycles, Eevee, Workbench
    \item Ustawienia w panelu właściwości (\key{Shift F7})
\end{itemize}

\section{Panele właściwości}
Dostęp przez \key{Shift F1} do \key{Shift F12}:
\begin{itemize}[itemsep=2pt]
    \item Render - wybór silnika renderowania
    \item Output - ustawienia wyjścia renderowania
    \item World - właściwości środowiska/świata
    \item Material - edycja materiałów
    \item Texture - zarządzanie teksturami
\end{itemize}

\section{Podstawy animacji}
\begin{itemize}[itemsep=2pt]
    \item \key{I} - wstawienie klatki kluczowej
    \item Klatki kluczowe zapisują stan właściwości obiektów
    \item Blender interpoluje wartości między klatkami
\end{itemize}

\vspace{1em}
\hrule
\vspace{0.5em}
\textit{Uwaga: ŚKM = Środkowy Klawisz Myszy (rolka)}

