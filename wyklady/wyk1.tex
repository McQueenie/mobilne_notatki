\section*{Wykład 1 - Przekaz multimedialny i platformy multimedialne}
\textbf{Data:} 3.03.2025

\section{Informacje organizacyjne}
\begin{itemize}[noitemsep]
    \item Kolokwium zaliczeniowe: 16.06.2025, forma elektroniczna
    \item Kolokwium poprawkowe: 23-26.06.2025
\end{itemize}

\section{Multimedia cyfrowe - definicja}
\textbf{Multimedia cyfrowe} to:
\begin{enumerate}[noitemsep]
    \item Element komunikacji
    \item Łączący minimum dwie formy: tekst, grafika, wideo, audio, animacje
    \item Posiadający format umożliwiający rozpowszechnianie
    \item Umożliwiający interakcję na urządzeniu cyfrowym
\end{enumerate}

\section{Cechy cyfrowych mediów wg Manovicha}
\begin{enumerate}[noitemsep]
    \item \textbf{Reprezentacja binarna i numeryczna} - dane zapisane cyfrowo
    \item \textbf{Strukturalna modułowość} - każdy obiekt zachowuje indywidualność
    \item \textbf{Automatyzacja} - uproszczenie tworzenia treści (szablony, skrypty)
    \item \textbf{Zmienność} - możliwość modyfikacji
    \item \textbf{Transkodowanie kulturowe} - przenoszenie między kontekstami
\end{enumerate}

\subsection{Edycja nieniszcząca}
Możliwość przywrócenia projektu do dowolnego stanu z historii zmian. Stosowana w formatach dedykowanych (np. .psd, .pptx).

\section{Przekaz multimedialny}
\textbf{Schemat komunikacji:}
\begin{center}
Nadawca $\rightarrow$ Kanał transmisyjny $\rightarrow$ Odbiorca
\end{center}

\textbf{Kluczowe elementy:}
\begin{itemize}[noitemsep]
    \item Źródło informacji
    \item Nadajnik (dostosowanie sygnału)
    \item Odbiornik (rekonstrukcja sygnału)
    \item Interpretator (użytkownik)
\end{itemize}

\section{Rodzaje danych multimedialnych}
\begin{enumerate}[noitemsep]
    \item Obrazy (pojedyncze, wideo, animacje)
    \item Dźwięk (mowa, muzyka, odgłosy)
    \item Grafika komputerowa (rastrowa, wektorowa)
    \item Teksty
    \item Dane mieszane i hybrydowe
    \item Metadane (opisy strumieni)
    \item Dane pomiarowe
    \item Instrukcje sterujące
    \item Warstwa synchronizacji
\end{enumerate}

\section{Informacja}
\textbf{Definicja:} To wszystko, co przekazane okazuje się użyteczne dla odbiorcy.

\textbf{Nośniki informacji} - sygnały dopasowane do:
\begin{itemize}[noitemsep]
    \item Charakteru treści
    \item Właściwości danych
    \item Natury opisywanego zjawiska
\end{itemize}

\section{Rejestracja danych}
\textbf{Kluczowe aspekty:}
\begin{itemize}[noitemsep]
    \item Fizyczne podstawy zjawisk (np. efekt fotoelektryczny)
    \item Zasady uzyskania sygnałów cyfrowych
    \item Zapewnienie wysokiej jakości i wierności zapisu
\end{itemize}

\subsection{Cyfrowe rejestratory obrazów}
\begin{itemize}[noitemsep]
    \item CCD (Charge-Coupled Device) - matryce fotoczułych komórek
    \item Parametry: czułość widmowa, czułość świetlna, zdolność rozdzielcza
\end{itemize}

\section{Prezentacja multimediów}
\subsection{Technologie wyświetlania}
\begin{enumerate}[noitemsep]
    \item \textbf{Elektroluminescencja} - monitory CRT
    \item \textbf{Wyładowanie jarzeniowe} - monitory plazmowe
    \item \textbf{Efekt ciekłych kryształów} - LCD
    \item \textbf{OLED} - diody organiczne (większa skala barw, wysoki kontrast)
\end{enumerate}

\subsection{Formy wizualizacji}
\begin{itemize}[noitemsep]
    \item Statyczna (ilustracje, wykresy)
    \item Dynamiczna (wideo)
    \item Komputerowa (interaktywna)
\end{itemize}

\section{Ewolucja Internetu}
\begin{itemize}[noitemsep]
    \item \textbf{Web 1.0} - statyczne strony, jednostronna komunikacja
    \item \textbf{Web 2.0} - interaktywność, media społecznościowe
    \item \textbf{Web 3.0} - semantyczny, sztuczna inteligencja
\end{itemize}

\section{Projektowanie}
\subsection{Wireframe}
Dwuwymiarowa reprezentacja określająca:
\begin{itemize}[noitemsep]
    \item Wykorzystanie przestrzeni
    \item Funkcjonalności interfejsu
    \item Hierarchię treści
    \item Połączenia między widokami
\end{itemize}

\subsection{Projektowanie uniwersalne}
7 zasad:
\begin{enumerate}[noitemsep]
    \item Równy dostęp
    \item Elastyczność użytkowania
    \item Prostota i intuicyjność
    \item Czytelna informacja
    \item Tolerancja na błędy
    \item Minimalizowanie wysiłku fizycznego
    \item Odpowiednie parametry przestrzeni
\end{enumerate}

\section{Dostępność cyfrowa}
Niewykluczanie żadnej osoby (w tym z niepełnosprawnościami) z możliwości korzystania z produktu cyfrowego.

\section{Standardy kompresji}
\begin{itemize}[noitemsep]
    \item \textbf{H.265 (HEVC)} - High Efficiency Video Coding
    \item Wspiera rozdzielczości do 8K (8192×4320)
    \item 2x lepsza kompresja niż H.264
\end{itemize}

