\section*{Wykład 9 - Animacje 2D}
\textit{5.05.2025 - Dr inż. Jakub Długosz}

\section{Definicja animacji}
\textbf{Animacja} -- technika filmowa polegająca na tworzeniu efektu ożywienia martwych kształtów przez dokonywanie serii pojedynczych zdjęć rysunków, wycinanek, kukiełek lub sylwetek i wyświetlaniu ich w sposób ciągły.

\textit{Kluczowa zasada Material Design}: \textbf{Ruch tworzy znaczenie} -- animacje są nieodzownym elementem komunikacji z użytkownikiem, tworzą iluzję obcowania z fizycznymi obiektami.

\section{12 zasad klasycznych animacji (Johnston i Thomas, 1981)}
Animatorzy Disney sformułowali fundamentalne zasady tworzenia animacji w książce \textit{The Illusion of Life: Disney Animation}. Zasady te są nadal aktualne w projektowaniu interfejsów.

\section{Wytyczne projektowania animacji}

\subsection{Apple Human Interface Guidelines - Motion}
\textbf{Najważniejsze zasady:}
\begin{enumerate}[itemsep=0pt]
    \item Używaj subtelnego ruchu do komunikacji
    \item Dodawaj animacje celowo, wspierając doświadczenie bez przytłaczania
    \item Uczyń animacje opcjonalnymi
    \item Dąż do realizmu i wiarygodności
    \item Preferuj krótkie, precyzyjne animacje
    \item Unikaj animacji w często występujących interakcjach
    \item Używaj animowanych symboli tam, gdzie ma to sens
\end{enumerate}

\textbf{Dostępność:}
\begin{itemize}[itemsep=0pt]
    \item Pozwól użytkownikom kontrolować efekty ruchu
    \item Zachowaj ostrożność przy elementach migających lub poruszających się
\end{itemize}

\subsection{Material Design - Motion}
Material Design definiuje precyzyjne wytyczne dla animacji, włączając:
\begin{itemize}[itemsep=0pt]
    \item \textbf{Transition patterns} -- wzorce przejść między stanami
    \item \textbf{Easing functions} -- funkcje kontroli przebiegu animacji
    \item \textbf{Duration tokens} -- tokeny czasowe dla różnych typów animacji
\end{itemize}

\section{Funkcje kontroli przebiegu animacji}

\subsection{Krzywe Béziera}
Funkcje easing oparte są na krzywych Béziera, które definiują przebieg animacji w czasie. Podstawowe typy:
\begin{itemize}[itemsep=0pt]
    \item \texttt{linear} -- stała prędkość
    \item \texttt{ease-in} -- powolny start, szybkie zakończenie
    \item \texttt{ease-out} -- szybki start, powolne zakończenie
    \item \texttt{ease-in-out} -- powolny start i koniec
\end{itemize}

\section{Implementacja animacji}

\subsection{CSS}
\begin{verbatim}
/* Przykład animacji CSS */
@keyframes slide {
    from { transform: translateX(0); }
    to { transform: translateX(100px); }
}
.element {
    animation: slide 0.3s ease-out;
}
\end{verbatim}

\subsection{Android}
\textbf{Podstawowe transformacje:}
\begin{itemize}[itemsep=0pt]
    \item \texttt{<scale>} -- przeskalowanie
    \item \texttt{<rotate>} -- obrót
    \item \texttt{<alpha>} -- przezroczystość
    \item \texttt{<translate>} -- przesunięcie
\end{itemize}

\textbf{Jetpack Compose} oferuje nowoczesne API do animacji z deklaratywnym podejściem.

\section{Specjalne efekty animacji}

\subsection{Efekt Kena Burnsa}
Technika animacji polegająca na powolnym przybliżaniu lub oddalaniu oraz panoramowaniu statycznego obrazu, często używana w prezentacjach fotografii.

\subsection{Efekt paralaksy}
Technika, w której elementy tła przesuwają się wolniej niż elementy pierwszoplanowe, tworząc iluzję głębi.

\section{Najlepsze praktyki}
\begin{enumerate}
    \item \textbf{Celowość} -- każda animacja powinna mieć jasny cel komunikacyjny
    \item \textbf{Wydajność} -- animacje nie powinny wpływać negatywnie na responsywność
    \item \textbf{Dostępność} -- możliwość wyłączenia animacji dla użytkowników wrażliwych na ruch
    \item \textbf{Spójność} -- zachowanie jednolitego stylu animacji w całej aplikacji
    \item \textbf{Mikroanimacje} -- małe, subtelne animacje poprawiające UX
\end{enumerate}

\section{Narzędzia}
\begin{itemize}[itemsep=0pt]
    \item \textbf{Figma} -- prototypowanie z animacjami przejść
    \item \textbf{CSS/JavaScript} -- implementacja w aplikacjach webowych
    \item \textbf{Android Animation Framework} -- natywne animacje Android
    \item \textbf{Core Animation (iOS)} -- framework Apple dla animacji
\end{itemize}

