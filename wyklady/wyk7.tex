\section*{Wykład 7 - Aplikacje multimedialne}
\textit{14.04.2025 - Notatki na podstawie wykładu dr inż. Jakuba Długosza}

\section{Myślenie projektowe (Design Thinking)}
\textbf{Design thinking} to metodologia rozwiązywania problemów skoncentrowana na użytkowniku.

\subsection{Mapa empatii}
Przedstawia produkt z perspektywy użytkownika w 4 kategoriach:
\begin{itemize}[noitemsep]
    \item Co użytkownik \textbf{myśli} i \textbf{czuje}?
    \item Co \textbf{słyszy}?
    \item Co \textbf{widzi}?
    \item Co \textbf{mówi} i \textbf{robi}?
\end{itemize}

\subsection{Tablica Kanban}
\textbf{Kanban} (jap. "sygnał wizualny") - metodyka oparta na dwóch zasadach:
\begin{enumerate}[noitemsep]
    \item Ograniczenie prac w toku (limity WIP)
    \item Wizualizacja pracy
\end{enumerate}

\textbf{5 elementów tablicy Kanban:}
\begin{itemize}[noitemsep]
    \item Sygnały wizualne
    \item Kolumny
    \item Limity prac w toku (WIP)
    \item Punkt zobowiązania
    \item Punkt dostarczenia
\end{itemize}

\section{Użytkownicy aplikacji multimedialnej}
\textbf{Charakterystyka:}
\begin{itemize}[noitemsep]
    \item Odbiorca masowy i anonimowy
    \item Klasyfikacja: kategorie demograficzne i psychograficzne
    \item Typy użytkowników:
    \begin{itemize}[noitemsep]
        \item \textbf{Subskrybenci} - zainteresowani regularnymi informacjami
        \item \textbf{Fani} - najbardziej oddani produktowi
    \end{itemize}
    \item \textbf{Grupa docelowa} - potencjalni użytkownicy produktu
\end{itemize}

\textbf{Użytkownicy vs Publiczność:} Użytkownicy podkreślają aktywną rolę, wchodzą w interakcje i kształtują produkt.

\section{Model 3-ech P}
\begin{enumerate}[noitemsep]
    \item \textbf{Preprodukcja} - planowanie i przygotowanie
    \item \textbf{Produkcja} - realizacja
    \item \textbf{Postprodukcja} - obróbka i finalizacja
\end{enumerate}

\section{Komunikacja wizualna}
\textbf{Definicja:} Przekazywanie informacji za pomocą symboli i form wizualnych.

\textbf{Kluczowe aspekty:}
\begin{itemize}[noitemsep]
    \item Koncentracja na percepcji przekazu (20\% widzenie/słyszenie, 80\% procesy kognitywne)
    \item Zasada Gestalt: "Całość jest większa od sumy wszystkich części"
\end{itemize}

\subsection{Diagram Gutenberga}
Opisuje naturalny wzorzec skanowania wzrokiem treści przez użytkownika.

\section{Zasady projektowania wizualnego}

\subsection{Zasada jedności}
Elementy muszą stanowić harmonijną całość poprzez:
\begin{itemize}[noitemsep]
    \item Wyrównanie
    \item Bliskość
    \item Podobieństwo
    \item Powtórzenia
\end{itemize}

\subsection{Zasada wyróżnienia}
Hierarchia wizualna realizowana przez:
\begin{itemize}[noitemsep]
    \item Skalę
    \item Kontrast
    \item Głębię
    \item Proporcje
    \item Ułożenie elementów
    \item Wykorzystanie przestrzeni
    \item Efekty graficzne
    \item Ikony, piktogramy
\end{itemize}

\subsection{Zasada trójpodziału}
Kompozycja fotograficzna oparta na podziale kadru na 9 równych części (3x3).

\subsection{Zasada psychologicznego domknięcia}
Przykłady:
\begin{itemize}[noitemsep]
    \item Trójkąt Kanizsy
    \item Sześcian Neckera
    \item Sześcian Eschera
\end{itemize}

\section{Trójkąt ekspozycji}
Określa zależności między:
\begin{itemize}[noitemsep]
    \item \textbf{Czasem naświetlania} (migawka)
    \item \textbf{Przysłoną}
    \item \textbf{Czułością ISO}
\end{itemize}

\textbf{EV (Exposure Value)} - jednostka miary ekspozycji:
\begin{itemize}[noitemsep]
    \item 0 EV = poprawna ekspozycja
    \item Zmiana o 1 EV = 2x więcej/mniej światła
\end{itemize}

\section{Paradygmaty aplikacji mobilnych}

\subsection{Sytuacyjność komunikacji}
\begin{itemize}[noitemsep]
    \item Brak ograniczeń czasowych i miejscowych
    \item Komunikacja permanentna
    \item Użytkowanie podyktowane konkretnymi celami
    \item Charakter utylitarny, wymagający uwagi
\end{itemize}

\subsection{Personalizacja}
Typy personalizacji:
\begin{itemize}[noitemsep]
    \item \textbf{Interfejsu} - dostosowanie wyglądu
    \item \textbf{Funkcjonalności} - wybór funkcji
    \item \textbf{Zawartości:}
    \begin{itemize}[noitemsep]
        \item Odbieranej (pobieranej)
        \item Tworzonej
    \end{itemize}
    \item \textbf{Strategii użytkowania}
\end{itemize}
Może być bezpłatna lub odpłatna.

\subsection{Innowacyjność}
\textbf{Atrybuty innowacji wg Everetta Rogersa:}
\begin{itemize}[noitemsep]
    \item Wyzwanie relacyjne
    \item Kompatybilność
    \item Obserwowalność
    \item Złożoność
    \item Testowalność
\end{itemize}

Dotyczy: technologii, modelu biznesowego, zawartości, dystrybucji i promocji.

\section{Własność intelektualna}
\textbf{Definicja:} Wytwory ludzkiego umysłu w materialnej postaci.

\textbf{Cechy praw własności intelektualnej:}
\begin{itemize}[noitemsep]
    \item Prawa wyłączne
    \item Ograniczone czasowo i terytorialnie
    \item Charakter materialny
    \item Zbywalne (z wyjątkiem osobistych praw autorskich)
\end{itemize}

\subsection{Licencje}
Przykłady popularnych licencji:
\begin{itemize}[noitemsep]
    \item GNU General Public License (GPL)
    \item Creative Commons (CC)
    \item MIT License
\end{itemize}

\section{Podsumowanie}
Aplikacje mobilne charakteryzują się trzema kluczowymi cechami: \textbf{sytuacyjnością}, \textbf{personalizacją} i \textbf{innowacyjnością}. Projektowanie wizualne opiera się na zasadach jedności i wyróżnienia, wykorzystując procesy kognitywne użytkownika. Ważne jest uwzględnienie praw własności intelektualnej i odpowiednie licencjonowanie treści.

