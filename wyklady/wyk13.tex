\section*{Wykład 13 - Blender - Animacje i Efekty}
\textbf{Data:} 2 czerwca 2025

\section{Animacje w Blenderze}

\textbf{Tworzenie klatek kluczowych:}
\begin{itemize}[leftmargin=*]
    \item \texttt{i} -- utworzenie klatki kluczowej z właściwościami zaznaczonych obiektów
    \item Klatki kluczowe przechowują stan obiektów (pozycja, rotacja, skala, właściwości materiałów)
    \item Blender automatycznie interpoluje wartości między klatkami kluczowymi
\end{itemize}

\section{Efekty specjalne - Dym i Ogień}

\subsection{Konfiguracja podstawowa}
\begin{enumerate}
    \item Dodanie obiektu emitującego (np. UV Sphere)
    \item Dodanie domeny (Cube otaczający emiter)
    \item Przypisanie fizyki:
    \begin{itemize}
        \item Emiter: \texttt{Fizyka → Ciecz → Flow}
        \item Domena: \texttt{Fizyka → Ciecz → Domain}
    \end{itemize}
\end{enumerate}

\subsection{Ustawienia domeny dymu}
\textbf{Kluczowe parametry:}
\begin{itemize}[leftmargin=*]
    \item \textbf{Adaptive Domain} -- automatyczne dostosowanie rozmiaru domeny
    \item \textbf{Resolution Divisions} -- zwiększenie do 256 dla lepszej jakości
    \item \textbf{Gas → Dissolve} -- rozpuszczanie dymu w czasie
\end{itemize}

\section{Edytor węzłów dla materiałów}

\subsection{Aktywacja}
Aby korzystać z edytora węzłów (\texttt{Shift + F3}):
\begin{itemize}[leftmargin=*]
    \item We właściwościach powierzchni materiału zaznaczyć \textbf{„Używaj węzłów"}
\end{itemize}

\subsection{Węzły dla efektów objętościowych}
\textbf{Podstawowa konfiguracja dla dymu:}
\begin{enumerate}
    \item \texttt{Attribute} (density) → \texttt{ColorRamp} → \texttt{Principled Volume}
    \item Połączenie z \texttt{Volume Output}
    \item Możliwość dodania emisji dla efektu ognia
\end{enumerate}

\section{Praktyczne wskazówki}

\begin{itemize}[leftmargin=*]
    \item Renderowanie animacji: \texttt{Ctrl + F12} (lub \texttt{Ctrl + Fn + F12})
    \item Symulacje wymagają cache -- przed renderowaniem należy przeliczyć fizykę
    \item Wyższa rozdzielczość = lepsza jakość, ale dłuższy czas obliczeń
    \item Adaptive Domain znacząco optymalizuje wydajność
\end{itemize}

\section{Zastosowania}
\begin{itemize}[leftmargin=*]
    \item Efekty specjalne w grach mobilnych
    \item Wizualizacje 3D w aplikacjach AR
    \item Renderowanie sekwencji dla cutscen
    \item Tworzenie assetów dla silników gier
\end{itemize}

